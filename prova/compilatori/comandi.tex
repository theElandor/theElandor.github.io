% Created 2023-06-03 sab 17:23
% Intended LaTeX compiler: pdflatex
\documentclass[11pt]{article}
\usepackage[utf8]{inputenc}
\usepackage[T1]{fontenc}
\usepackage{graphicx}
\usepackage{grffile}
\usepackage{longtable}
\usepackage{wrapfig}
\usepackage{rotating}
\usepackage[normalem]{ulem}
\usepackage{amsmath}
\usepackage{textcomp}
\usepackage{amssymb}
\usepackage{capt-of}
\usepackage{hyperref}
\author{Matteo Lugli}
\date{\today}
\title{\textbf{Commands cheatsheet}}
\hypersetup{
 pdfauthor={Matteo Lugli},
 pdftitle={\textbf{Commands cheatsheet}},
 pdfkeywords={},
 pdfsubject={},
 pdfcreator={Emacs 27.1 (Org mode 9.3)}, 
 pdflang={English}}
\begin{document}

\maketitle
\section{Handling PassManager}
\label{sec:orgc82caa7}
Con il comando \textbf{opt} si può alterare il comportamento del
pass manager di default, specificando manualmente quali 
sono i passi da svolgere. Si possono caricare dei passi
custom in formato .so.
Ecco delle direttive da usare all'interno del makefile per generare
il file ".so".
\begin{verbatim}
OPTIMIZER := libTestPass.so
OBJs := $(subst .cpp,.o,$(wildcard lib/*.cpp))

LLVM_VERSION ?= 14

CXXFLAGS := $(shell llvm-config-$(LLVM_VERSION) --cxxflags) -fPIC

all: $(OPTIMIZER) 

$(OPTIMIZER): $(OBJs)
	$(CXX) -dylib -shared $^ -o $@
\end{verbatim}

\noindent\rule{\textwidth}{0.5pt}
\subsection{Tutorial 01}
\label{sec:orgf7eb6e5}
\begin{verbatim}
.PHONY: clean
clean:
	$(RM) $(OPTIMIZER) $(OBJs)
load:
	opt -load-pass-plugin=./libTestPass.so -passes=test-pass $(file) -disable-output
\end{verbatim}
In questo caso per eseguire il passo è necessario dare in pasto al make anche
il file su cui eseguire l'ottimizzazione: 
\begin{verbatim}
make clean all load file=test/Loop.bc
\end{verbatim}

\noindent\rule{\textwidth}{0.5pt}
\subsection{Tutorial 02}
\label{sec:org59e8a41}
\begin{verbatim}
.PHONY: clean setup
clean:
	$(RM) $(OPTIMIZER) $(OBJs)
setup:
	opt -load-pass-plugin=./libLocalOpts.so -passes=algebraic $(firstrepr) -o $(bcfile)
	llvm-dis $(bcfile) -o $(optrepr)

.PHONY: auto
auto:
	make clean all setup firstrepr=./testTransform/Foo.ll bcfile=./testTransform/FooOpt.bc optrepr=./testTransform/FooOpt.ll
\end{verbatim}
\section{OMP}
\label{sec:org8c42ecc}
Comando per compilare un file c, linkando la libreria \textbf{openmp} e 
produrre un file intermedio (dopo la procedura di lowering).
\begin{verbatim}
gcc -fopenmp -fdump-tree-omplower example.c  
\end{verbatim}
\end{document}
